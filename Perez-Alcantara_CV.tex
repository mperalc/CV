% !TeX program = xelatex  

\documentclass[letterpaper]{deedy-resume} 
\usepackage{fixltx2e}
\usepackage{hyperref}
\usepackage{color}
\usepackage[document]{ragged2e}


\begin{document}
	
	%----------------------------------------------------------------------------------------
	%	TITLE SECTION
	%----------------------------------------------------------------------------------------
	
	\begin{flushleft}
		\Huge \textbf{Marta Pérez Alcántara}\\*
		\begin{small}
			\vspace{-7mm}
			\textbf{ \href{mailto:martac@well.ox.ac.uk}{\underline} martac@well.ox.ac.uk | +44 07895810750 
			}
		\end{small}
	\end{flushleft}
	
	%----------------------------------------------------------------------------------------
	%	LEFT COLUMN
	%----------------------------------------------------------------------------------------
	
	\begin{minipage}[t]{0.29\textwidth} 
		
		\section{Education} 
		
		\subsection{University of Oxford}
		\descript{DPhil Genomic Medicine and Statistics}
		\location{ 2015 - 2019 | awaiting \textit{viva} }
		
		\sectionspace 
		
		\subsection{University of}
		\subsection{Edinburgh}		
		\descript{MSc Quantitative Genetics and Genome Analysis}
		\location{2013-2014 | Distinction}
		
		\sectionspace 
		
		\subsection{Pablo de Olavide}
		\subsection{University}				
		\descript{5-year BSc Biotechnology}
		\location{2008-2013 | 8.2/10}
		
		\sectionspace 
		
		\sectionspace 
		
	
		
		\section{Selected Skills}
		
		\subsection{Programming}
		R \textbullet{} Python (including snakemake) \\
		Bash \textbullet{} Perl \textbullet{} Java  \textbullet{} LaTeX  \\
		R Markdown
		
		\sectionspace 
		
		\subsection{Bioinformatics}
		\textit{Omics} data analysis, including:\\
		\textbullet{} sequence mapping, feature calling and quantification pipelines \textbullet{} differential expression  \textbullet{} QTLs
		\textbullet{} functional analysis of GWAS variants
		
		\sectionspace 
				
		\subsection{Wet-lab}
		CRISPR KO and modulation \textbullet{} ChIP-seq \textbullet{} human cell culture \\
		iPSC differentiation \textbullet{} FACS \textbullet{} qPCR
		
		\sectionspace
		
		\subsection{Languages}
		English (Advanced, C2) \textbullet{} Spanish (Native)
		\textbullet{} French (Basic, A2) \textbullet{} German (Basic, A1)
		\sectionspace 
		\sectionspace
		
		\section{Links} 
		Github:// \href{https://github.com/mperalc}{\textbf{mperalc}} \\
		LinkedIn://  \href{https://www.linkedin.com/in/marta-p\%C3\%A9rez-alc\%C3\%A1ntara-51171861}{\textbf{marta-pérez-alcántara}} \\
		
		\sectionspace
	\end{minipage} 
	\hfill
	%
	%----------------------------------------------------------------------------------------
	%	RIGHT COLUMN
	%----------------------------------------------------------------------------------------
	%
	\begin{minipage}[t]{0.66\textwidth} 
		
		\section{Postgraduate projects}
		
		\runsubsection{DPhil} |\descript{\small Understanding the role of islet development in type 2 diabetes susceptibility
		}
		\vspace{\topsep} 
		\begin{tightitemize}
			\item Supervisors: Prof. Mark McCarthy and Prof. Ben Davies.\\
			\item Investigated how pancreatic islet development is involved in T2D susceptibility, through the analysis of an hiPSC differentiation model of beta cells. \textit{Omics} data (RNA-seq, ATAC-seq, DNA methylation and H3K27ac ChIP-seq) was used to characterize temporal patterns and networks of gene expression and DNA regulation, to identify important genes regulating islet development and T2D risk. Data integration highlighted candidate T2D genes whose function in development was analysed in the laboratory, using CRISPR to modulate their expression and assessing how this altered the differentiation process.
			
		\end{tightitemize}
		
		\sectionspace 
		
		\runsubsection{MSc} |\descript{\small Parent-of-origin effects in human complex trait variation}
		\begin{tightitemize}
			\item Dissertation project supervised by Prof Chris S. Haley. \\
			\item I evaluated the role of imprinting in the variation of obesity measurements (WHR and BMI), investigating its weight in heritability and incorporating it in genome-wide association analyses.
		\end{tightitemize}
	
			\sectionspace 
	
	\section{Publications} 
	
	\textbullet{} \textbf{Perez-Alcantara M}, Honoré C, Wesolowska-Andersen A, et al. Diabetologia (2018). Patterns of differential gene expression in a cellular model of human islet development, and relationship to type 2 diabetes predisposition.\href{https://doi.org/10.1007/s00125-018-4612-4}{https://doi.org/10.1007/s00125-018-4612-4}\\
	\sectionspace
	\textbullet{} Gascoyne DM, Spearman H, Lyne L, Puliyadi R, \textbf{Perez-Alcantara M}, et al. (2015) The Forkhead Transcription Factor FOXP2 Is Required for Regulation of p21$^{WAF1/CIP1}$ in 143B Osteosarcoma Cell Growth Arrest. \href{https://doi.org/10.1371/journal.pone.0128513}{PLOS ONE 10(6): e0128513.}
		
		\sectionspace 
		
		
		
		\section{Conference presentations} 
	    \textbullet{}\textbf{Talk:} "Chromatin accessibility patterns of a hiPSC model of islet development highlight type 2 diabetes risk loci in beta cell differentiation" \\
	    {\footnotesize Oct 2019 | American Society of Human Genetics | Houston, USA\\}
	    \textbullet{}\textbf{Poster:} "Chromatin accessibility patterns of a hiPSC model of islet development and type 2 diabetes risk" \\
	    {\footnotesize April 2019 | EASD Islet Study Group and Beta-Cell Workshop | Oxford, UK\\}
	    \textbullet{}\textbf{Poster:} "Chromatin accessibility patterns of a hiPSC model of islet development and type 2 diabetes risk" \\
	    {\footnotesize Oct 2018 | American Society of Human Genetics | San Diego, USA\\}
	    \textbullet{}\textbf{Talk:} "Human islet differentiation model highlights developmental mechanisms contributing to type 2 diabetes pathology" \\
	    {\footnotesize March 2018 | Leena Peltonen School of Human Genomics | Les Diablerets, Switzerland\\}
	    \textbullet{}\textbf{Talk:} "Human islet differentiation model highlights developmental mechanisms contributing to type 2 diabetes pathology" \\
	    {\footnotesize Feb 2018 | Keystone Frontiers in Islet Biology and Diabetes | Keystone, USA\\}
		\textbullet{}\textbf{Talk:} "Transcriptomic profiling of the developing human islet and mechanisms of type 2 diabetes predisposition" \\
		{\footnotesize Oct 2017 | American Society of Human Genetics | Orlando, USA\\}

		\sectionspace 
		\sectionspace
		\sectionspace
		\sectionspace
		\sectionspace
	\end{minipage}
	\hfill
%
%	%----------------------------------------------------------------------------------------
%	%	TITLE SECTION
%	%----------------------------------------------------------------------------------------
	
	\begin{flushleft}
		\Huge \textbf{Marta Pérez Alcántara}\\*
		\begin{small}
			\vspace{-7mm}
			\textbf{ \href{mailto:martac@well.ox.ac.uk}{\underline} martac@well.ox.ac.uk | +44 07895810750 
			}
		\end{small}
	\end{flushleft}
	
	%----------------------------------------------------------------------------------------
%	%	LEFT COLUMN
%	%----------------------------------------------------------------------------------------
	
	\begin{minipage}[t]{0.99\textwidth} 
	\section{Grants and awards}
		
		\textbullet{}\location{\color{darkgray} Epstein Trainee Award Semifinalist | American Society of Human Genetics}
		{\footnotesize Oct 2019\\}
		\textbullet{}\location{\color{darkgray} Wellcome Trust PhD Studentship}
		{\footnotesize Oct 2015 - Oct 2019\\}
		\textbullet{}\location{\color{darkgray} Bronze medal in European semifinals of MIT iGEM synthetic biology competition}
		{\footnotesize Oct 2011\\}
		
		\sectionspace 
		
	\section{Other research experience}
	
	\descript{Genetics Department | University of Seville}
	
	\location{Research intern | 2014-2015 }
	Supervised by Prof. E. Cerda-Olmedo. I studied the role of carotenoids and terpenoids in the genetic mechanisms of sexual reproduction of the fungus \textit{Phycomyces}.\\
	\smallsectionspace
	
	\descript{University of Oxford}
	
	\location{Research intern | Summer 2013}
	Supervised by Prof. A. Banham and Dr D. Gascoyne. Analysed the function of transcription factors FOXP1 and FOXP2 in breast cancer.\\
	\smallsectionspace
	
	\descript{University of Cardiff}
	
	\location{Research intern | Summer 2012}
	Supervised by Prof. J.P. Aggleton and Dr S. Vann. Characterized the patterns of gene expression in the brain during mice memory fixation.\\
	\smallsectionspace
	
	\descript{iGEM competition | University of Seville}
	
	\location{ Team member | 2011}
	Participant in the MIT synthetic biology competition. Genetically engineered \textit{E. coli} strains to employ them as logic gates.
	\sectionspace 


\section{Other activities}

 \textbullet{} Organiser of \textbf{postgraduate student presentation sessions}: October 2015 - October 2016
 
  \textbullet{} Elected member of Amnesty International Spain, International Affairs Committee: May 2013 - May 2015
  
  \textbullet{} Coordinator of Amnesty International local volunteer group, Seville branch: October 2014 - August 2015
 
\sectionspace 
\end{minipage} 
\end{document}